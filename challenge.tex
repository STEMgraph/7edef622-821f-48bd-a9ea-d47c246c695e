\learningobjective{At the end of this challenge, the scholar will be able to explain the concept of 	exttt{std::vector} as a dynamic, resizable container in C++, including its advantages over manually implemented dynamic arrays in C and its use cases for efficient data manipulation.}
\begin{challenge}
    \chatitle{Dynamic Resizable Containers with std::vector in C++}
    \begin{chadescription}
    In C++, the \texttt{std::vector} class represents a dynamic array that can resize itself automatically, providing a powerful and flexible alternative to fixed-size arrays. 
    Unlike \texttt{std::array} or raw C-style arrays, \texttt{std::vector} dynamically allocates memory as needed, enabling it to grow or shrink while maintaining efficient element access.

    The \texttt{std::vector} container is especially useful for situations where the number of elements is unknown at compile time or may change frequently. 
    With built-in functions like \texttt{push_back()}, \texttt{pop_back()}, and \texttt{insert()}, \texttt{std::vector} simplifies the process of managing dynamic data. 
    Furthermore, it provides support for random access and seamless integration with modern C++ algorithms, making it a highly versatile container.

    In this challenge, you will explore how to use \texttt{std::vector} for dynamic storage and manipulation of data. 
    You will also compare it to a manually implemented dynamic array in C to understand the benefits of using this STL container. 
    By the end, you will have a solid understanding of when and why to use \texttt{std::vector} in C++ programs.
    \end{chadescription}

    \begin{task}
        Write a simple program using \texttt{std::vector} that performs the following steps:
        \begin{enumerate}
            \item Declare an empty \texttt{std::vector} of integers.
            \item Use \texttt{push_back()} to add the numbers \texttt{10}, \texttt{20}, and \texttt{30} to the vector.
            \item Print all elements of the vector using a range-based \texttt{for} loop.
            \item Remove the last element of the vector using \texttt{pop_back()}.
            \item Print the updated vector to the terminal.
        \end{enumerate}

        Save your program and run it to ensure the output matches your expectations.

        \begin{questions}
            \item How does \texttt{push_back()} simplify adding elements to a dynamic array compared to manually managing memory in C?
            \item What happens when you call \texttt{pop_back()} on an empty vector? How can you prevent errors in such cases?
            \item Why is it more efficient to add elements to the end of a vector than to insert them at arbitrary positions?
        \end{questions}
    \end{task}

    \begin{task}
        Write a program using \texttt{std::vector} that performs the following steps:
        \begin{enumerate}
            \item Declare and initialize an \texttt{std::vector} with the values \texttt{1, 2, 3, 4, 5}.
            \item Implement a function named \texttt{calculateAverage} that takes the \texttt{std::vector} as a parameter and returns the average of its elements.
            \item Insert the value \texttt{6} at the second position of the vector.
            \item Remove the first element of the vector.
            \item Print the vector after each modification and display the calculated average.
        \end{enumerate}

        Save your program, compile it, and ensure it runs correctly.

        \begin{questions}
            \item How does using \texttt{insert()} and \texttt{erase()} in \texttt{std::vector} differ from modifying a manually implemented dynamic array in C?
            \item What advantages does the \texttt{calculateAverage} function gain from \texttt{std::vector}'s size-tracking capabilities compared to manually tracking array size in C?
            \item How does the \texttt{std::vector}'s ability to grow dynamically simplify the overall program design?
        \end{questions}
    \end{task}

    \begin{task}
        Write a program using \texttt{std::vector} that performs the following steps:
        \begin{enumerate}
            \item Declare and initialize an \texttt{std::vector} of integers with the values \texttt{3, 1, 4, 1, 5, 9, 2}.
            \item Implement a function named \texttt{findSecondLargest} that returns the second largest element in the vector.
            \item Manually reverse the elements of the vector by swapping elements from the beginning and the end.
            \item Add the value \texttt{7} to the end of the vector and recalculate the second largest element.
            \item Print the reversed vector and the second largest element before and after adding the new value.
        \end{enumerate}

        Save your program, compile it, and ensure it runs correctly. This task requires careful use of indexing to manipulate the vector dynamically.

        \begin{questions}
            \item How does using \texttt{std::vector} for dynamic indexing and manipulation compare to manually allocating and resizing arrays in C?
            \item What challenges arise when reversing the vector manually, and how does \texttt{std::vector}'s size-tracking help address these challenges?
            \item Why is the \texttt{findSecondLargest} function easier to implement with \texttt{std::vector} than with a raw array?
        \end{questions}
    \end{task}

    \begin{advise}
        Working with \texttt{std::vector} helps you understand the power of dynamic memory management in C++ without the complexity of manual allocation and resizing in C. 
        Unlike raw arrays or \texttt{std::array}, \texttt{std::vector} provides built-in methods for managing data, resizing, and accessing elements efficiently. 
        Through this challenge, you have explored how \texttt{std::vector} simplifies dynamic operations, such as adding, removing, and rearranging elements. 
        Always remember to use \texttt{std::vector} when you need random access and dynamic resizing, as it provides the best of both worlds for many common use cases. 
        With practice, you'll find \texttt{std::vector} to be one of the most versatile and frequently used containers in C++ programming.
    \end{advise}
\end{challenge}
